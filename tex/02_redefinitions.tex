%%% У даному файлі визначайте всі необхідні вам нові команди TeX
%%% або робіть перевизначення існуючих, наприклад...

% Перевизначення символу порожньої множини та знаків "більше-дорівнює", "менше-дорівнює" на прийняті у нас
\let\oldemptyset\emptyset
\let\emptyset\varnothing
\let\geq\geqslant
\let\leq\leqslant

% Визначення нових математичних команд
% \newcommand*{\binsp}[1]{\ensuremath \left\{0, 1\right\}^{#1}}       % {0, 1}^m
% \newcommand*{\xor}{\ensuremath \oplus}                              % \xor = (+)
% \newcommand*{\GF}[1]{\ensuremath \mathbb F_{#1}}                    % F_n
% \newcommand*{\GFgroup}[1]{\ensuremath \mathbb F^{*}_{#1}}           % F^*_n
% \newcommand*{\Zring}[1]{\ensuremath \mathbb Z_{#1}}                 % Z_n
% \newcommand*{\Zgroup}[1]{\ensuremath \mathbb Z^{*}_{#1}}            % Z^*_n
% \newcommand*{\Jset}[1]{\ensuremath \mathbb J_{#1}}                  % J_n
% \newcommand*{\Qset}[1]{\ensuremath \mathbb Q_{#1}}                  % Q_n
% \newcommand*{\PQset}[1]{\ensuremath \widetilde{\mathbb Q}_{#1}}     % Q~_n
% \newcommand*{\cyclic}[1]{\ensuremath \left\langle {#1} \right\rangle}                  % <g>
% \newcommand*{\Legendre}[2]{\ensuremath \left( \frac{#1}{#2} \right)}  % символ Лежандра/Якоби
% \newcommand*{\compinv}[1]{\ensuremath {#1}^{\left\langle -1 \right\rangle}}  % обратный по композиции

% Інший спосіб визначення математичного оператору
% \DeclareMathOperator{\ord}{ord}
% \DeclareMathOperator{\lcm}{lcm}
% \DeclareMathOperator{\Li}{Li}
% \DeclareMathOperator{\Coef}{Coef}
% \DeclareMathOperator{\Log}{Log}
% \DeclareMathOperator{\Exp}{Exp}
% \DeclareMathOperator{\Res}{Res}
% \DeclareMathOperator{\charact}{char}
% \DeclareMathOperator{\Sym}{Sym}

%%%%%%%%%%%%%%%%%%%%%%%%%%%%%%%%%%%%%%%%%%%%%%%%%%%%%%%%%%%%%%%%%%%%%%%%%%%%%%%%%%%%%%%%%%%%%%%%%%%%%%%%

\newcommand\N{\ensuremath{\mathbb{N}}}
\newcommand\R{\ensuremath{\mathbb{R}}}
\newcommand\Z{\ensuremath{\mathbb{Z}}}
\renewcommand\O{\ensuremath{\emptyset}}
\newcommand\Q{\ensuremath{\mathbb{Q}}}

% Put x \to \infty below \lim
\let\svlim\lim\def\lim{\svlim\limits}

%Make implies and impliedby shorter
\let\implies\Rightarrow
\let\impliedby\Leftarrow
\let\iff\Leftrightarrow
\let\epsilon\varepsilon

% hide parts
\newcommand\hide[1]{}


% Environments
\makeatother
% For box around Definition, Theorem, \ldots
\usepackage{mdframed}
\mdfsetup{skipabove=1em,skipbelow=0em}
\theoremstyle{definition}
% \newmdtheoremenv[nobreak=true]{definition}{Definition}
\newmdtheoremenv[nobreak=true]{characteristic}{Characteristic}
% \newmdtheoremenv[nobreak=true]{corollary}{Corollary}
% \newmdtheoremenv[nobreak=true]{lemma}{Lemma}
% \newmdtheoremenv[nobreak=true]{proposition}{Proposition} % твердження
\newmdtheoremenv[nobreak=true]{law}{Law}
\newmdtheoremenv[nobreak=true]{postulate}{Postulate}
\newtheorem*{consequence}{Consequence}
\newtheorem*{practical}{Practical}
\newtheorem*{terminology}{Terminology}
% \newtheorem*{example}{Example}

% \newmdtheoremenv[nobreak=true]{definition}{Definition}
\newtheorem*{eg}{Example}
\newtheorem*{notation}{Notation}
\newtheorem*{previouslyseen}{As previously seen}
% \newtheorem*{remark}{Remark}
% \newtheorem*{problem}{Problem}
\newtheorem*{observe}{Observe}
\newtheorem*{property}{Property}
\newtheorem*{intuition}{Intuition}
\newmdtheoremenv[nobreak=true]{prop}{Proposition}
% \newmdtheoremenv[nobreak=true]{theorem}{Theorem}
% \newmdtheoremenv[nobreak=true]{corollary}{Corollary}

% End example and intermezzo environments with a small diamond (just like proof
% environments end with a small square)
\usepackage{etoolbox}
\AtEndEnvironment{vb}{\null\hfill$\diamond$}%
\AtEndEnvironment{intermezzo}{\null\hfill$\diamond$}%

% \newcommand\todo[1]{\textcolor{red}{TODO: #1}}


