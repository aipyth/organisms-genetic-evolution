%!TEX root = ../thesis.tex
% створюємо вступ
\textbf{Актуальність дослідження.} 
Актуальність даного дослідження полягає
створенні унікальної моделі розвитку організмів з
використанням генетичних алгоритмів та глибинного навчання.
Дана модель робить внесок у сферу обчислювальної біології та штучного життя,
надаючи модульну та гнучку модель, яка надає потенціал у розумінні
фундаментальних принципів еволюції, адаптації та стратегій виживання.
Моделюючи механізми еволюції, ця робота може дати уявляення про механізми
адаптації та еволюції організмів у відповідь на подразники навколишнього середовища.
Також використання нейронних мереж у розвитку організмів дає можливість
досліджувати ефекти різних стратегій навчання та вирішення проблем.
Це може мати значні наслідки для галузі штучного інтелекту, 
зокрема для розробки більш ефективних та дієвих алгоритмів навчання.

% основа досліджень еволюційної динаміки 

% потенціал для розуміння фундаментальних принципів еволюції, адаптації та стратегій виживання

% робить внесок у сферу обчислювальної біології та штучного життя, надаючи модульну та гнучку модель

% потенційне застосування в різних галузях, таких як біоінформатика, машинне навчання та робототехніка, де можуть бути застосовані принципи природної еволюції та адаптації.

% Фундаментальна біологія та еволюційна теорія: 
% Ваші дослідження поглиблюють наше розуміння еволюційної біології. 
% Моделюючи механізми еволюції, ваша робота може дати уявлення про те, 
% як організми адаптуються та еволюціонують у відповідь на навколишнє середовище.

% Штучний інтелект і машинне навчання: 
% Вивчаючи вплив складності структур нейронних мереж на розвиток організмів, 
% ви ефективно досліджуєте ефекти різних стратегій навчання та вирішення проблем. 

% у тому, що без нього ви не одержите диплом про вищу освіту. Відповідно, ви повинні 
% оформити результати вашого дослідження належним чином.

% Вступ є однією із самих формалізованих частин дипломної роботи. На початку 
% ви у двох-трьох абзацах повинні окреслити проблематику та актуальність 
% вашого дослідження, після чого переходити до мети та завдання.

\textbf{Метою дослідження} є
аналіз адаптації та еволюційний розвиток модельованих організмів 
за допомогою генетичних алгоритмів та глибинного навчання.
% на основі розробленого програмного забезпечення.
% є певна абстрактна недосяжна річ на кшталт 
% загальнолюдського щастя на горизонті. Для досягнення мети необхідно 
% розв'язати \textbf{задачу дослідження}, яка полягає у чомусь суттєво більш 
% конкретному. Для розв'язання задачі необхідно вирішити такі завдання:
Для досягнення мети, потрібно розробити гнучку та потужну модель,
що ставить такі завдання:

\begin{enumerate}
\item провести огляд опублікованих джерел за тематикою дослідження;
% \item на основі отриманого огляду джерел за тематикою дослідження
%   провести формування вимог до моделі;
\item розробити та реалізувати модель;
\item провести аналіз результатів роботи моделі, поведінки організмів у середовищі
  та факторів впливу на еволюційний процес.
\end{enumerate}

\emph{Об'єктом дослідження} є еволюційний розвиток простих організмів 
за допомогою генетичних алгоритмів та глибинного навчання.

\emph{Предметом дослідження} є симуляція еволюції простих організмів 
у двовимірному обмеженому непервному просторі.

При розв’язанні поставлених завдань використовувались такі \emph{методи дослідження}:
спостереження, порівняння,
методи лінійної алгебри, теорії ймовірностей, математичної статистики,
методи комп'ютерного моделювання.

% і тут коротенький перелік (наприклад, але не обмежуючись: методи лінійної та абстрактної 
% алгебри, теорії імовірностей, математичної статистики, комбінаторного 
% аналізу, теорії кодування, теорії складності алгоритмів, методи 
% комп’ютерного та статистичного моделювання) 

\textbf{Наукова новизна} отриманих результатів полягає
застосуванні моделі розвитку організмів, яка є симбіозом
нейронних мереж та генетичних алгоритмів,
для дослідженні спливу різних структур нейронних мереж 
в організмах та параметрів моделі,
що дає глибше зрозуміти взаємозв'язок між складністю механізмів 
навчання та ефективністю виживання.
Дослідження проливає світло на феномен формування груп серед організмів 
--- аспект, який часто ігнорується в простіших моделях еволюції. 
Визнання цієї складної поведінки навіть у простих організмів привносить 
новий вимір реалізму в еволюційні дослідження.

\hide{
... -- тут необхідно 
перелічити, що саме нового з точки зору науки несе ваша робота. До усіх 
тверджень, які сюди виносяться, подумки (а іноді й явним чином) потрібно 
ставити слово <<вперше>> -- і ці твердження повинні залишатись істинними.
}



\textbf{Практичне значення} результатів полягає в
отриманні гнучкої моделі еволюції популяції організмів та
подальшому дослідженні впливу структури організмів на їх адаптацію.
Розроблена бібліотека спрямована на вивчення адаптації організмів, 
надаючи універсальний інструмент для досліджень в галузі еволюційної біології. 
Вона дозволяє легко задавати параметри, 
сприяючи ефективному експериментуванню та швидкій перевірці гіпотез.
Дана робота слугує інструментом в дослідженні динаміки еволюції.

\hide{
... -- тут необхідно 
зазначити практичну користь від результатів вашої роботи. Що саме можна 
покращити, підвищити (або знизити), зробити гарного (або уникнути 
поганого) після вашого дослідження.
}


% \textbf{Апробація результатів та публікації.} Наприкінці вступу необхідно 
% зазначити перелік конференцій, семінарів та публікацій, в яких викладено 
% результати вашої роботи. Якщо результати вашої роботи ніде не 
% доповідались, опускайте даний абзац.
