%!TEX root = ../thesis.tex
% створюємо перелік умовних позначень, скорочень і термінів

\textbf{індивід} --- потенційне рішення

\textbf{популяція} --- набір потенційних рішень

\textbf{генотип або геном} --- структура даних особини, що використовується під час селекції 

\textbf{хромосома} --- генотип у вигляді векторного гена фіксованої довжини 

\textbf{ген} --- певна позиція слота в хромосомі 

\textbf{фенотип} --- фізичні характеристики особини, її дії під час оцінки придатності 

% \textbf{генерація} --- один цикл оцінки придатності, селекції та збирання популяції


\textit{
(Якщо ви не використовуєте перелік умовних позначень, просто приберіть 
даний розділ.)}

\textit{(БУДЬ ЛАСКА, ПРОСЛІДКУЙТЕ, ЩОБ НОМЕР СТОРІНКИ СПІВПАДАВ ІЗ СПРАВЖНІМ! Це залежить від того, наскільки великим є ваш зміст.
Номер сторінки проставляється у файлі thesis.tex, рядок 35.)
}
