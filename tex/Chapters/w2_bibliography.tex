%!TEX root = ../thesis.tex
% створюємо список використаної літератури
\begin{thebibliography}    
    \bibitem{sad} 
    asda

    \bibitem{dca}
    asd [Електронний ресурс]. --- Режим доступу: \url{dsf}.

@inproceedings{debSelfadaptiveSimulatedBinary2007,
  title = {Self-Adaptive Simulated Binary Crossover for Real-Parameter Optimization},
  author = {Deb, Kalyan and Sindhya, Karthik and Okabe, Tatsuya},
  date = {2007-07-07},
  pages = {1187--1194},
  doi = {10.1145/1276958.1277190},
  abstract = {Simulated binary crossover (SBX) is a real-parameter re-combination operator which is commonly used in the evo-lutionary algorithm (EA) literature. The operator involves a parameter which dictates the spread of o spring solutions vis-a-vis that of the parent solutions. In all applications of SBX so far, researchers have kept a fixed value throughout a simulation run. In this paper, we suggest a self-adaptive procedure of updating the parameter so as to allow a smooth navigation over the function landscape with iteration. Some basic principles of classical optimization literature are uti-lized for this purpose. The resulting EAs are found to produce remarkable and much better results compared to the original operator having a fixed value of the parameter. Studies on both single and multiple objective optimization problems are made with success.},
  file = {/home/john/Zotero/storage/8ZWLQJBQ/Deb et al. - 2007 - Self-adaptive simulated binary crossover for real-.pdf}
}

@article{debSIMULATEDBINARYCROSSOVER,
  title = {{{SIMULATED BINARY CROSSOVER FOR CONTINUOUS SEARCH SPACE}}},
  author = {Deb, Kalyanmoy and Agrawal, Ram Bhusan},
  abstract = {The success of binary-coded genetic algorithms (GAs) in problems having discrete search space largely depends on the coding used to represent the problem variables and on the crossover operator that propagates building-blocks from parent strings to children strings. In solving optimization problems having continuous search space, binary-coded GAs discretize the search space by using a coding of the problem variables in binary strings. However, the coding of real-valued variables in nite-length strings causes a number of di culties|inability to achieve arbitrary precision in the obtained solution, xed mapping of problem variables, inherent Hamming cli problem associated with the binary coding, and processing of Holland's schemata in continuous search space. Although, a number of real-coded GAs are developed to solve optimization problems having a continuous search space, the search powers of these crossover operators are not adequate. In this paper, the search power of a crossover operator is de ned in terms of the probability of creating an arbitrary child solution from a given pair of parent solutions. Motivated by the success of binary-coded GAs in discrete search space problems, we develop a real-coded crossover (we called the simulated binary crossover or SBX) operator whose search power is similar to that of the single-point crossover used in binary-coded GAs. Simulation results on a number of real-valued test problems of varying di culty and dimensionality suggest that the real-coded GAs with the SBX operator are able to perform as good or better than binary-coded GAs with the single-point crossover. SBX is found to be particularly useful in problems having multiple optimal solutions with a narrow global basin and in problems where the lower and upper bounds of the global optimum are not known a priori. Further, a simulation on a two-variable blocked function shows that the real-coded GA with SBX works as suggested by Goldberg and in most cases the performance of real-coded GA with SBX is similar to that of binary GAs with a single-point crossover. Based on these encouraging results, this paper suggests a number of extensions to the present study.},
  langid = {english},
  file = {/home/john/Zotero/storage/BHXTXV72/Deb and Agrawal - SIMULATED BINARY CROSSOVER FOR CONTINUOUS SEARCH S.pdf}
}

@article{guptaEmbodiedIntelligenceLearning2021,
  title = {Embodied Intelligence via Learning and Evolution},
  author = {Gupta, Agrim and Savarese, Silvio and Ganguli, Surya and Fei-Fei, Li},
  date = {2021-10-06},
  journaltitle = {Nature Communications},
  shortjournal = {Nat Commun},
  volume = {12},
  number = {1},
  pages = {5721},
  publisher = {{Nature Publishing Group}},
  issn = {2041-1723},
  doi = {10.1038/s41467-021-25874-z},
  url = {https://www.nature.com/articles/s41467-021-25874-z},
  urldate = {2023-06-01},
  abstract = {The intertwined processes of learning and evolution in complex environmental niches have resulted in a remarkable diversity of morphological forms. Moreover, many aspects of animal intelligence are deeply embodied in these evolved morphologies. However, the principles governing relations between environmental complexity, evolved morphology, and the learnability of intelligent control, remain elusive, because performing large-scale in silico experiments on evolution and learning is challenging. Here, we introduce Deep Evolutionary Reinforcement Learning (DERL): a computational framework which can evolve diverse agent morphologies to learn challenging locomotion and manipulation tasks in complex environments. Leveraging DERL we demonstrate several relations between environmental complexity, morphological intelligence and the learnability of control. First, environmental complexity fosters the evolution of morphological intelligence as quantified by the ability of a morphology to facilitate the learning of novel tasks. Second, we demonstrate a morphological Baldwin effect i.e., in our simulations evolution rapidly selects morphologies that learn faster, thereby enabling behaviors learned late in the lifetime of early ancestors to be expressed early in the descendants lifetime. Third, we suggest a mechanistic basis for the above relationships through the evolution of morphologies that are more physically stable and energy efficient, and can therefore facilitate learning and control.},
  issue = {1},
  langid = {english},
  keywords = {Computational science,Computer science},
  file = {/home/john/Zotero/storage/GGKC5I2I/Gupta et al. - 2021 - Embodied intelligence via learning and evolution.pdf}
}

@article{koraCrossoverOperatorsGenetic2017,
  title = {Crossover {{Operators}} in {{Genetic Algorithms}}: {{A Review}}},
  shorttitle = {Crossover {{Operators}} in {{Genetic Algorithms}}},
  author = {Kora, Padmavathi and Yadlapalli, Priyanka},
  date = {2017-03-15},
  journaltitle = {International Journal of Computer Applications},
  shortjournal = {International Journal of Computer Applications},
  volume = {162},
  pages = {34--36},
  doi = {10.5120/ijca2017913370},
  abstract = {Genetic Algorithms are the population based search and optimization technique that mimic the process of natural evolution. Genetic algorithms are very effective way of finding a very effective way of quickly finding a reasonable solution to a complex problem. Performance of genetic algorithms mainly depends on type of genetic operators which involve crossover and mutation operators. Different crossover and mutation operators exist to solve the problem that involves large population size. Example of such a problem is travelling sales man problem, which is having a large set of solution. In this paper we will discuss different crossover operators that help in solving the problem.},
  file = {/home/john/Zotero/storage/AZPSU7ZQ/Kora and Yadlapalli - 2017 - Crossover Operators in Genetic Algorithms A Revie.pdf}
}

@book{lukeEssentialsMetaheuristicsSet2013,
  title = {Essentials of Metaheuristics: A Set of Undergraduate Lecture Notes; {{Online Version}} 2.0},
  shorttitle = {Essentials of Metaheuristics},
  author = {Luke, Sean},
  date = {2013},
  edition = {2. ed},
  publisher = {{Lulu}},
  location = {{S.l.}},
  isbn = {978-1-300-54962-8},
  langid = {english},
  pagetotal = {239},
  file = {/home/john/Zotero/storage/ANTGDR5L/Luke - 2013 - Essentials of metaheuristics a set of undergradua.pdf}
}

@article{mccallGeneticAlgorithmsModelling2005,
  title = {Genetic Algorithms for Modelling and Optimisation},
  author = {McCall, John},
  date = {2005-12-01},
  journaltitle = {Journal of Computational and Applied Mathematics},
  shortjournal = {Journal of Computational and Applied Mathematics},
  series = {Special {{Issue}} on {{Mathematics Applied}} to {{Immunology}}},
  volume = {184},
  number = {1},
  pages = {205--222},
  issn = {0377-0427},
  doi = {10.1016/j.cam.2004.07.034},
  url = {https://www.sciencedirect.com/science/article/pii/S0377042705000774},
  urldate = {2023-05-31},
  abstract = {Genetic algorithms (GAs) are a heuristic search and optimisation technique inspired by natural evolution. They have been successfully applied to a wide range of real-world problems of significant complexity. This paper is intended as an introduction to GAs aimed at immunologists and mathematicians interested in immunology. We describe how to construct a GA and the main strands of GA theory before speculatively identifying possible applications of GAs to the study of immunology. An illustrative example of using a GA for a medical optimal control problem is provided. The paper also includes a brief account of the related area of artificial immune systems.},
  langid = {english},
  keywords = {Evolution,Genetic algorithms,Immunology,Optimisation},
  file = {/home/john/Zotero/storage/LNEEAURX/McCall - 2005 - Genetic algorithms for modelling and optimisation.pdf;/home/john/Zotero/storage/ZJ9EEBW8/S0377042705000774.html}
}

@online{rooyEvolvingSimpleOrganisms,
  title = {Evolving {{Simple Organisms}} Using a {{Genetic Algorithm}} and {{Deep Learning}} from {{Scratch}} with {{Python}}},
  author = {Rooy, Nathan},
  url = {https://nathanrooy.github.io/posts/2017-11-30/evolving-simple-organisms-using-a-genetic-algorithm-and-deep-learning/},
  urldate = {2023-05-31},
  abstract = {Nathan A. Rooy | nathanrooy.github.io},
  langid = {english},
  organization = {{nathanrooy.github.io}},
  file = {/home/john/Zotero/storage/QJGTLJST/evolving-simple-organisms-using-a-genetic-algorithm-and-deep-learning.html}
}

@online{roudenkoDominanceBasedCrossover2005,
  title = {Dominance {{Based Crossover Operator}} for {{Evolutionary Multi-objective Algorithms}}},
  author = {Roudenko, Olga and Schoenauer, Marc},
  date = {2005-05-29},
  eprint = {cs/0505080},
  eprinttype = {arxiv},
  doi = {10.48550/arXiv.cs/0505080},
  url = {http://arxiv.org/abs/cs/0505080},
  urldate = {2023-06-01},
  abstract = {In spite of the recent quick growth of the Evolutionary Multi-objective Optimization (EMO) research field, there has been few trials to adapt the general variation operators to the particular context of the quest for the Pareto-optimal set. The only exceptions are some mating restrictions that take in account the distance between the potential mates - but contradictory conclusions have been reported. This paper introduces a particular mating restriction for Evolutionary Multi-objective Algorithms, based on the Pareto dominance relation: the partner of a non-dominated individual will be preferably chosen among the individuals of the population that it dominates. Coupled with the BLX crossover operator, two different ways of generating offspring are proposed. This recombination scheme is validated within the well-known NSGA-II framework on three bi-objective benchmark problems and one real-world bi-objective constrained optimization problem. An acceleration of the progress of the population toward the Pareto set is observed on all problems.},
  pubstate = {preprint},
  keywords = {Computer Science - Artificial Intelligence,Mathematics - Numerical Analysis},
  file = {/home/john/Zotero/storage/IXGAS4DZ/Roudenko and Schoenauer - 2005 - Dominance Based Crossover Operator for Evolutionar.pdf;/home/john/Zotero/storage/5UQ9HD6C/0505080.html}
}

@online{sloss2019EvolutionaryAlgorithms2019,
  title = {2019 {{Evolutionary Algorithms Review}}},
  author = {Sloss, Andrew N. and Gustafson, Steven},
  date = {2019-06-03},
  eprint = {1906.08870},
  eprinttype = {arxiv},
  eprintclass = {cs},
  doi = {10.48550/arXiv.1906.08870},
  url = {http://arxiv.org/abs/1906.08870},
  urldate = {2023-06-02},
  abstract = {Evolutionary algorithm research and applications began over 50 years ago. Like other artificial intelligence techniques, evolutionary algorithms will likely see increased use and development due to the increased availability of computation, more robust and available open source software libraries, and the increasing demand for artificial intelligence techniques. As these techniques become more adopted and capable, it is the right time to take a perspective of their ability to integrate into society and the human processes they intend to augment. In this review, we explore a new taxonomy of evolutionary algorithms and resulting classifications that look at five main areas: the ability to manage the control of the environment with limiters, the ability to explain and repeat the search process, the ability to understand input and output causality within a solution, the ability to manage algorithm bias due to data or user design, and lastly, the ability to add corrective measures. These areas are motivated by today's pressures on industry to conform to both societies concerns and new government regulatory rules. As many reviews of evolutionary algorithms exist, after motivating this new taxonomy, we briefly classify a broad range of algorithms and identify areas of future research.},
  pubstate = {preprint},
  keywords = {Computer Science - Machine Learning,Computer Science - Neural and Evolutionary Computing},
  file = {/home/john/Zotero/storage/ZIBSIRPP/Sloss and Gustafson - 2019 - 2019 Evolutionary Algorithms Review.pdf;/home/john/Zotero/storage/VU9CDNM8/1906.html}
}

@article{slowikEvolutionaryAlgorithmsTheir2020,
  title = {Evolutionary Algorithms and Their Applications to Engineering Problems},
  author = {Slowik, Adam and Kwasnicka, Halina},
  date = {2020-08-01},
  journaltitle = {Neural Computing and Applications},
  shortjournal = {Neural Comput \& Applic},
  volume = {32},
  number = {16},
  pages = {12363--12379},
  issn = {1433-3058},
  doi = {10.1007/s00521-020-04832-8},
  url = {https://doi.org/10.1007/s00521-020-04832-8},
  urldate = {2023-06-03},
  abstract = {The main focus of this paper is on the family of evolutionary algorithms and their real-life applications. We present the following algorithms: genetic algorithms, genetic programming, differential evolution, evolution strategies, and evolutionary programming. Each technique is presented in the pseudo-code form, which can be used for its easy implementation in any programming language. We present the main properties of each algorithm described in this paper. We also show many state-of-the-art practical applications and modifications of the early evolutionary methods. The open research issues are indicated for the family of evolutionary algorithms.},
  langid = {english},
  keywords = {Differential evolution,Evolution strategy,Evolutionary programming,Genetic algorithm,Genetic programming,Nature-inspired methods,Real-life applications},
  file = {/home/john/Zotero/storage/JHKR7F34/Slowik and Kwasnicka - 2020 - Evolutionary algorithms and their applications to .pdf}
}

@article{xiongOptimizingLongTermBank2020,
  title = {Optimizing {{Long-Term Bank Financial Products Portfolio Problems}} with a {{Multiobjective Evolutionary Approach}}},
  author = {Xiong, Jian and Zhang, Chao and Kou, Gang and Wang, Rui and Ishibuchi, Hisao and Alsaadi, Fawaz E.},
  date = {2020-04-07},
  journaltitle = {Complexity},
  volume = {2020},
  pages = {e3106097},
  publisher = {{Hindawi}},
  issn = {1076-2787},
  doi = {10.1155/2020/3106097},
  url = {https://www.hindawi.com/journals/complexity/2020/3106097/},
  urldate = {2023-06-01},
  abstract = {With the development of economy, the requirement of financial planning for individuals or families is emerging. In the era of the Internet, individual investors can conveniently enter the market and purchase financial products. Traditional portfolio management models focus on risky markets such as stock markets. However, risk-averse investors, such as normal families, may concern appropriate long-term financial planning. This paper considers the problem of portfolio management of bank financial products with a long-term planning horizon. By taking into account the final return and the flexibility, a multiobjective model of long-term portfolio is proposed. A multiobjective evolutionary approach is employed for the handling of conflicting objectives. Test instances are generated to illustrate the problem. Experiment results show that the presented algorithm can efficiently find trade-off solutions. Our experimental results also show that crossover probabilities should be separately implemented for long-term portfolio problems with hybrid encoding. Performance comparison of different crossover operators suggest that, for a real-valued encoding part, the simulated binary crossover (SBX) has a better performance than BLX- operator. While for a binary encoding part, a uniform crossover operator might be appropriate for large-scale instances. The proposed multiobjective model in this paper provides risk-averse investors with an appropriate decision support model for the long-term financial planning and management.},
  langid = {english},
  file = {/home/john/Zotero/storage/DX35FYM7/Xiong et al. - 2020 - Optimizing Long-Term Bank Financial Products Portf.pdf}
}

@book{yuIntroductionEvolutionaryAlgorithms2010,
  title = {Introduction to {{Evolutionary Algorithms}}},
  author = {Yu, Xinjie and Gen, Mitsuo},
  editorb = {Roy, Rajkumar},
  editorbtype = {redactor},
  date = {2010},
  series = {Decision {{Engineering}}},
  volume = {0},
  publisher = {{Springer London}},
  location = {{London}},
  doi = {10.1007/978-1-84996-129-5},
  url = {http://link.springer.com/10.1007/978-1-84996-129-5},
  urldate = {2023-06-04},
  isbn = {978-1-84996-128-8 978-1-84996-129-5},
  langid = {english},
  file = {/home/john/Zotero/storage/VN2RBZJG/Yu and Gen - 2010 - Introduction to Evolutionary Algorithms.pdf}
}
 

\end{thebibliography}
