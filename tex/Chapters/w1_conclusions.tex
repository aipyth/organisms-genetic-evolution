%!TEX root = ../thesis.tex
% створюємо Висновки до всієї роботи

% Загальні висновки до роботи повинні підсумовувати усі ваші досягнення у 
% даному напрямку досліджень.

% За кожним пунктом завдань, поставлених у вступі, у висновках повинен 
% міститись звіт про виконання: виконано, не виконано, виконано частково (І 
% чому саме так). Наприклад, якщо першим поставленим завданням у вас іде 
% <<огляд літератури за тематикою досліджень>>, то на початку висновків ви 
% повинні зазначити, що <<у ході даної роботи був проведений аналіз 
% опублікованих джерел за тематикою (...), який показав, що (...)>>. Окрім 
% простої констатації про виконання ви повинні навести, які саме результати 
% ви одержали та проінтерпретувати їх з точки зору поставленої задачі, мети 
% та загальної проблематики.

\todo[inline]{висновки про поставленні завдання}

% В ідеалі загальні висновки повинні збиратись з висновків до кожного 
% розділу, але ідеал недосяжний. :) Однак висновки не повинні містити 
% формул, таблиць та рисунків. Дозволяється (та навіть вітається) 
% використовувати числа (на кшталт <<розроблена методика дозволяє підвищити 
% ефективність пустопорожньої балаканини на $2.71\%$>>).

% Наприкінці висновків необхідно зазначити напрямки подальших досліджень: 
% куди саме, як вам вважається, необхідно прямувати наступним дослідникам у 
% даній тематиці.




% Ця робота забезпечила поглиблене вивчення моделі на 
% основі генетичного алгоритму,
% яка імітує еволюцію організмів у віртуальному середовищі. 

\todo[inline]{
- чи відображає модель з достатньою точністю поведінку реальних систем? (так, навести приклади, якщо зможете)

Модель репрезентує просту абстракцію реального механізму розвитку організмів,
але не є повним еквівалентом, хоча і поділяє кілька основних концепцій.
Дана модель наслідує механізми виживання та кількості енергії:
реальні біологічні організми споживають певні ресурси, що надають їм можливість
виживати. Цей принцип є фундаментальним у такій моделі.
Подібно до справжньої біологічної еволюції, 
модель використовує генетичні алгоритми для імітації концепцій 
природного відбору, мутацій та успадкування. 
Нейронна мережа є абстрактною картиною того,
як реальні істоти використовують свій генетичний код
для створення складних дій та функцій.
Те, як нейронна мережа використовується
для відтворення процесу <<мислення>> організму,
схоже на те, як реальні тварини використовують
свій мозок для взаємодії з навколишнім середовищем.
Однак важливо зазначити, що спрощення та абстракції, 
необхідні для комп'ютерного моделювання означають, 
що дана модель не є ідеальним відображенням реального світу
через обмеження у: сенсорах, спрощеному середовищі,
відсутності росту, складності геному.



Дослідження виявило динамічний зв'язок між складністю геному, 
швидкістю еволюції та довгостроковою адаптацією. 
Підібрані гарні параметри для старту моделювання та механізми
еволюції.
Було показано, що організми з простішою генетичною архітектурою адаптуються 
швидше через менший простір рішень. 
Однак, незважаючи на більш повільну початкову еволюцію, 
індивіди зі складнішою архітектурою геному демонстрували кращий потенціал 
для довгострокової адаптації.
Цікаво, що, незважаючи на простоту моделі, було виявлено
зачатки розвитку груп у популяції. 

% Дослідження також висвітлило труднощі, пов'язані з дефіцитом 
% ресурсів, оскільки групи часто гинули через труднощі в отриманні їжі, 
% що підкреслює компроміс між соціальною поведінкою 
% та конкурентною боротьбою за ресурси. 
% Це узгоджується з тенденціями, що спостерігаються в біологічних системах, 
% і підкреслює необхідність додаткових досліджень еволюції кооперативних 
% або взаємовигідних дій у відповідь на екологічні обмеження.

% Результати дослідження демонструють потужність генетичних алгоритмів 
% у відтворенні складних біологічних процесів, 
% які є результатом простих еволюційних механізмів. 
% Проект підкреслює корисність еволюційних обчислень як інструменту 
% для розуміння біологічної еволюції та 
% вдосконалення складних систем штучного інтелекту.

% Загалом, це дослідження закладає основу для майбутніх 
% досліджень еволюційної динаміки, зокрема, 
% впливу змін умов навколишнього середовища та генетичної складності 
% на адаптацію та стратегії виживання організмів.


\todo[inline]{
- чи надає ускладнення форми організму йому переваги у життєдіяльності? (парадоксально, але не завжди - прості, але швидкі і прожерливі організми швидше накопичують вагу)

- чи спостерігаються паттерни групової свідомості (ройового штучного інтелекту) в поведінці групи? (про певні залежності можна говорити, але потрібні додаткові обчислювальні експерименти. якщо щось спостерігалося - можна вказати)
}

\todo[inline]{напрямки подальших досліджень}
