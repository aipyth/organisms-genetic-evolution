%!TEX root = ../thesis.tex
% створюємо Висновки до всієї роботи

\conclusions

% Ця робота забезпечила поглиблене вивчення моделі на 
% основі генетичного алгоритму,
% яка імітує еволюцію організмів у віртуальному середовищі. 

% Загальні висновки до роботи повинні підсумовувати усі ваші досягнення у 
% даному напрямку досліджень.

% За кожним пунктом завдань, поставлених у вступі, у висновках повинен 
% міститись звіт про виконання: виконано, не виконано, виконано частково (І 
% чому саме так). Наприклад, якщо першим поставленим завданням у вас іде 
% <<огляд літератури за тематикою досліджень>>, то на початку висновків ви 
% повинні зазначити, що <<у ході даної роботи був проведений аналіз 
% опублікованих джерел за тематикою (...), який показав, що (...)>>. Окрім 
% простої констатації про виконання ви повинні навести, які саме результати 
% ви одержали та проінтерпретувати їх з точки зору поставленої задачі, мети 
% та загальної проблематики.

% \todo[inline]{висновки про поставленні завдання}

Було проведено огляд по доступним джерелам по темі моделей розвитку
організмів, з яких було виведено вимоги до даної моделі.
Також огляд по джерелам еволюційних алгоритмів підштовхнув до розширення
загальності генетичного алгоритму, використаного у даній роботі.

У другому розділі було сформовано принцип роботи із нейронними мережами
у моделі та вимоги до побудови цієї моделі даної роботи.

У третьому розділі описано реалізацію моделі, а також оглянуто результати
її роботи. 

% В ідеалі загальні висновки повинні збиратись з висновків до кожного 
% розділу, але ідеал недосяжний. :) Однак висновки не повинні містити 
% формул, таблиць та рисунків. Дозволяється (та навіть вітається) 
% використовувати числа (на кшталт <<розроблена методика дозволяє підвищити 
% ефективність пустопорожньої балаканини на $2.71\%$>>).

% Наприкінці висновків необхідно зазначити напрямки подальших досліджень: 
% куди саме, як вам вважається, необхідно прямувати наступним дослідникам у 
% даній тематиці.





% \todo[inline]{
% - чи відображає модель з достатньою точністю поведінку реальних систем? (так, навести приклади, якщо зможете)
% }

Ця робота забезпечила поглиблене вивчення моделі на 
основі генетичного алгоритму,
яка імітує еволюцію організмів у віртуальному середовищі. 
Модель репрезентує просту абстракцію реального механізму розвитку організмів,
але не є повним еквівалентом, хоча і поділяє кілька основних концепцій.
Дана модель наслідує механізми виживання та кількості енергії:
реальні біологічні організми споживають певні ресурси, що надають їм можливість
виживати. Цей принцип є фундаментальним у такій моделі.
Подібно до справжньої біологічної еволюції, 
модель використовує генетичні алгоритми для імітації концепцій 
природного відбору, мутацій та успадкування. 
Нейронна мережа є абстрактною картиною того,
як реальні істоти використовують свій генетичний код
для створення складних дій та функцій.
Те, як нейронна мережа використовується
для відтворення процесу <<мислення>> організму,
схоже на те, як реальні тварини використовують
свій мозок для взаємодії з навколишнім середовищем.
Однак важливо зазначити, що спрощення та абстракції, 
необхідні для комп'ютерного моделювання означають, 
що дана модель не є ідеальним відображенням реального світу
через обмеження у: сенсорах, спрощеному середовищі,
відсутності росту, складності геному.



% Згідно з дослідженням, існує постійний зв’язок між складністю геному, швидкістю еволюції та довгостроковою адаптацією. 
% % Хороші параметри для початку моделювання та еволюції
% Організми з простішою генетичною архітектурою були показані здатними адаптуватися швидше через менший простір рішень. 
% Незважаючи на більш повільну початкову еволюцію, люди зі складнішою структурою геному були більш адаптованими в довгостроковій перспективі.
% Цікаво, що незважаючи на простоту моделі, було виявлено початки розвитку груп у популяції. 

% Дослідження виявило динамічний зв'язок між складністю геному, 
% швидкістю еволюції та довгостроковою адаптацією. 
% Підібрані гарні параметри для старту моделювання та механізми
% еволюції.
% Було показано, що організми з простішою генетичною архітектурою адаптуються 
% швидше через менший простір рішень. 
% Однак, незважаючи на більш повільну початкову еволюцію, 
% індивіди зі складнішою архітектурою геному демонстрували кращий потенціал 
% для довгострокової адаптації.
% Цікаво, що, незважаючи на простоту моделі, було виявлено
% зачатки розвитку груп у популяції. 

% Дослідження також висвітлило труднощі, пов'язані з дефіцитом 
% ресурсів, оскільки групи часто гинули через труднощі в отриманні їжі, 
% що підкреслює компроміс між соціальною поведінкою 
% та конкурентною боротьбою за ресурси. 
% Це узгоджується з тенденціями, що спостерігаються в біологічних системах, 
% і підкреслює необхідність додаткових досліджень еволюції кооперативних 
% або взаємовигідних дій у відповідь на екологічні обмеження.

Результати дослідження демонструють потужність генетичних алгоритмів 
у відтворенні складних біологічних процесів, 
які є результатом простих еволюційних механізмів. 
Проект підкреслює корисність еволюційних обчислень як інструменту 
для розуміння біологічної еволюції та 
вдосконалення складних систем штучного інтелекту.



% \todo[inline]{
% - чи надає ускладнення форми організму йому переваги у життєдіяльності? (парадоксально, але не завжди - прості, але швидкі і прожерливі організми швидше накопичують вагу)
% }

% Згідно з дослідженням, існує динамічний зв'язок між складністю геному, 
% швидкістю еволюції та довготривалою адаптацією. 
% Було продемонстровано, що організми з простішою генетичною 
% архітектурою адаптуються швидше, оскільки простір рішень є меншим. 
% Однак особини зі складнішою архітектурою геному мають 
% вищий потенціал для довготривалої адаптації, 
% незважаючи на повільнішу еволюцію на початковому етапі.

Дослідження розкриває складні взаємозв'язки між складністю геному 
організму, швидкістю еволюції та здатністю до довготривалої адаптації 
до навколишнього середовища. 
Отримані дані свідчать про те, 
що організми з більш простою генетичною структурою здатні 
швидше адаптуватися. 
Менший простір рішень уможливлює швидший відбір і 
застосування вигідних генетичних особливостей, 
що може бути причиною такої швидкості. 
Нижчий рівень складності таких геномів полегшує організмам 
швидке розпізнавання та набуття ознак, 
які допоможуть їм вижити в поточному середовищі, 
що сприяє швидшій еволюції.

Це дослідження також показує, 
що складніша геномна архітектура дає значну перевагу з 
точки зору потенціалу довгострокової адаптивності, 
навіть якщо спочатку вона розвивається повільніше. 
Складний геном має більше змінних, 
що розширює простір пошуку генетичного алгоритму 
і збільшує ймовірність виявлення різних якостей, 
які можуть бути корисними. 
Ці складнощі дають більший простір для адаптації, 
збільшуючи можливості цих видів реагувати на навколишнє середовище 
і пристосовуватися до нього, 
навіть якщо вони спочатку сповільнюють темп еволюції.

% Результати дослідження привертають увагу до потенційного конфлікту 
% між здатністю до довгострокової еволюції та 
% короткостроковою адаптивністю. 
% Істоти зі складнішими генетичними системами мають більшу 
% здатність до довгострокової адаптації, 
% навіть якщо простіші істоти можуть процвітати на ранніх стадіях 
% завдяки своїй здатності швидко реагувати на нагальні екологічні обмеження.
% Незважаючи на повільніший темп початкової еволюції, 
% ці складні істоти можуть перевершити своїх простіших побратимів у 
% довгостроковій перспективі завдяки постійному використанню широкої 
% зони генетичного пошуку та здатності пристосовуватися до 
% різноманітних мінливих умов навколишнього середовища.

Ці відкриття суттєво вплинуть на наші знання про еволюцію, 
адаптацію та роль складності геному в цих процесах. 
Це спонукає нас переосмислити еволюцію як процес, 
що врівноважує короткострокові вигоди швидкої адаптації 
з довгостроковими перевагами генетичного різноманіття 
і складності, а не як просту боротьбу за виживання.

% Згідно з дослідженням, існує постійний зв’язок між складністю геному, 
% швидкістю еволюції та довгостроковою адаптацією. 
% Хороші параметри для початку моделювання та еволюції
% організми з простішою генетичною архітектурою були 
% показані здатними адаптуватися швидше через менший простір рішень. 
% Незважаючи на більш повільну початкову еволюцію, 
% особи зі складнішою структурою геному були більш адаптованими 
% в довгостроковій перспективі.

% \todo[inline]{
% - чи спостерігаються паттерни групової свідомості (ройового штучного інтелекту) в поведінці групи? (про певні залежності можна говорити, але потрібні додаткові обчислювальні експерименти. якщо щось спостерігалося - можна вказати)
% % }

Цікаво, що незважаючи на простоту моделі, 
було виявлено початки розвитку груп у популяції. 
Під час експерименту було спостережено за 
розвитком організмів у малих групах людей, 
які рухалися за схожим шаблоном.
Часто саме наявність подібного організму поруч 
мотивувала змінити курс і слідувати за ним.
Це дає переконливі емпіричні докази групової поведінки, 
навіть якщо ці угрупування не надають переваг у 
такому конкурентному середовищі.
Такі групи часто гинули, оскільки пошук їжі в 
такому бідному середовищі вимагає змагання з усіма, 
навіть із членами групи.

% \todo[inline]{напрямки подальших досліджень}

Розроблена тут модель є відносно простою з точки зору 
складності організмів і навколишнього середовища. 
Майбутні дослідження можуть включати розробку більш складних 
організмів з розширеними можливостями або включення 
додаткових факторів навколишнього середовища. 
Це дозволило б вивчати більш складні взаємодії та поведінку.

Виникнення групової поведінки в певних ситуаціях 
у даній моделі є цікавим напрямком для майбутніх досліджень. 
Можна дослідити умови, за яких виникає групова поведінка, 
і її вплив на виживання та пристосованість організмів.

Хоча вивчено деякі ефекти зміни складності нейронної мережі, 
що використовується організмами, існує потенціал для дослідження 
більш досконалих архітектур нейронних мереж. 
Це включає використання згорткових або рекурентних нейронних мереж, 
які можуть запропонувати організмам різні можливості.

У поточній моделі середовище є статичним. 
Майбутні дослідження можуть дозволити середовищу 
еволюціонувати разом з організмами, 
що призведе до гонки між адаптацією та мінливими умовами.

Загалом, це дослідження закладає основу для майбутніх 
досліджень еволюційної динаміки, зокрема, 
впливу змін умов навколишнього середовища та генетичної складності 
на адаптацію та стратегії виживання організмів.
