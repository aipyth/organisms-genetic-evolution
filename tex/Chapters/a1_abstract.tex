%!TEX root = ../abstract.tex

\abstractUkr

\hide{
Кваліфікаційна робота містить: ??? стор., ??? рисунки, ??? таблиць, ??? джерел.

У рефераті роботи ви повинні коротко (два-три абзаци) викласти, що саме 
було зроблено у цій роботі. Перші три речення реферату (після статистичних 
даних) повинні окреслити мету роботи, об'єкт та предмет дослідження. Після 
цього викладаються основні результати, одержані в ході дослідження.

Наприкінці анотації великими літерами зазначаються ключові слова. Ось так:

% наприкінці анотації потрібно зазначити ключові слова
\MakeUppercase{КЛЮЧОВІ СЛОВА, СИМЕТРИЧНА КРИПТОГРАФІЯ, ФІЗТЕХ НАЙКРАЩІЙ}
}


%%%% Рішенням кафедри з 2018 року ми прибираємо анотації російською мовою
% \abstractRus
%
%Русская аннотация должна быть точным переводом украинской (включая 
%статистику и ключевые слова).

\abstractEng

The English abstract must be the exact translation of the Ukrainian 
``annotation'' (including statistical data and keywords).

% Не прибирайте даний рядок
\clearpage
