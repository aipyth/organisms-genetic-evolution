%!TEX root = ../thesis.tex
% створюємо розділ
\chapter{Застосування глибинного навчання у моделюванні розвитку простих організмів}
\label{chap:theory}

Проіллюструємо симбіотичний зв'язок між нейронними 
мережами та генетичними алгоритмами у моделюванні 
еволюції видів в екосистемі. 
Спочатку ми розглянемо нейронні мережі, 
які є обчислювальними системами, 
натхненними біологічними нейронними мережами, 
що складають мозок тварин. 
Ці мережі в даному контексті слугують мозком 
наших модельованих організмів, 
отримуючи інформацію про навколишнє середовище, 
обробляючи її та обираючи відповідні реакції.

Обрали генетичні алгоритми замість традиційних підходів до навчання, 
таких як градієнтний спуск. 
Кожен індивід в нашій популяції представляється
набором ваг та зсувів нейронної мережі, 
які функціонують як ДНК та фенотип організму. 
Ця нова ідея генетичного представлення нейронних мереж лежить 
в основі наших зусиль, а ефективність мережі у 
реагуванні на стимули навколишнього середовища слугує 
ключовим показником придатності для кожного індивіда.

Дана модель намагається імітувати процеси адаптації, 
виживання та еволюції, використовуючи принципи глибокого 
навчання та еволюційні алгоритми. 
Мета --- створити надійну модель еволюції організмів у часі, 
яка враховує не лише пристосованість окремих організмів, 
але й динаміку екосистеми, в якій вони живуть. 


%%%%%%%%%%%%%%%%%%%%%%%%%%%%%%%%%%%%%%%%%%%%%%%%%%%%%%%%%%%%%%%%%%%%%%%%%%%%%%%
\section{Нейронні мережі}

Штучні нейронні мережі створені за подібністю до людського мозку, 
який еволюціонував протягом мільйонів років.
Нейронна мережа складається з багатьох взяємопов'язаних нод, 
які представляють собою штучні нейрони. 
Всі ноди зі зв'язками між ними утворюють одну 
структуру обробки інформації.

Ці мережі є частиною ширшого предмету 
машинного навчання і досягають успіху у виявленні 
закономірностей та прогнозуванні на основі складних, 
багатовимірних даних.


Кожен нейрон отримує вхідні дані, 
множить їх на певні ваги, застосовує зсув, 
а потім пропускає через функцію активації. 
Математична операція нейрона може бути виражена 
наступним чином: 
\[ z = w x + b \]
де $x$ - вхідне значення, $w$ - вага, 
$b$ - зсув, і $z$ - зважена сума. 
Після отримання зваженої суми застосовується
функція активації $f$, результат якої є результатом роботи нейрону:
\[ a = f(z) \] 

Ваги та зсуви нейронної мережі є основними параметрами. 
Ваги визначають силу впливу вхідних даних на вихідні, 
тоді як зсуви дозволяють переміщати функцію активації по горизонталі. 

Функція активації нейрона визначає його вихід при 
наявності входу або комбінації входів. 
У нейронних мережах використовуються різні 
типи функцій активації.

\begin{enumerate}
  \item \textbf{Сигмоїдна функція} стискає вхідні 
    значення в діапазоні від 0 до 1. 
    Вона часто використовується в задачах бінарної класифікації. 
    Сигмоїдна функція активації має наступну формулу:
    \[ f(z) = \frac{1}{1 + e^{-z}} \] 
  \item \textbf{ReLU} (Rectified Linear Unit) стала більш вживаною функцією
    активації для нейронних мереж, оскільки при ній мережа навчаються
    швидше та зазвичай видає кращі результати. Для позитивних значень повертається
    значення безпосередньо, а для негативних нуль.
    \[ f(z) = \max(0, z) \]
  \item \textbf{Гіперболічна функція} також видає результат у певній межі, а саме в межах від
    -1 до 1.
    Зазвичай вона використовується саме в прихованих шарах нейронних мереж.
    \[ f(z) = \frac{e^{z} - e^{-z}}{e^{z} + e^{-z}} \] 

\end{enumerate}

Кожен вузол в шарі пов'язаний з кожним вузлом в наступному шарі. 
Цим зв'язкам присвоюються ваги, які змінюються в міру того, 
як мережа навчається на основі даних.
Вхідний рівень є початковим рівнем та приймає вхідні дані.
Останній шар - це вихідний шар, який відповідає за отримання кінцевого результату. 
Один або декілька прихованих шарів виконують обчислення і перетворення даних між ними.

\begin{figure}[ht]
    \centering
    \incfig{приклад-структури-нейронної-мережі}
    \caption{Приклад структури нейронної мережі}
    \label{fig:приклад-структури-нейронної-мережі}
\end{figure}

Щоб ефективно обчислювати вихід нейронної мережі, 
ми векторизуємо обчислення для нейронів у шарі. 
Назвемо вагову матрицю $W$, вектор зсуву $b$, 
вхідний вектор $A_p$ та функцію активації $g$ для даного шару. 
Вихід шару, $A$, можна обчислити наступним чином:
\[ Z = W A_{p} + b \] 


Глибинне навчання --- це підгалузь машинного навчання, 
у якій застосовуються нейронні мережі із 
декількома прихованими шарами.
Такі мережі також відомі як "глибокі" нейронні мережі. 



%%%%%%%%%%%%%%%%%%%%%%%%%%%%%%%%%%%%%%%%%%%%%%%%%%%%%%%%%%%%%%%%%%%%%%%%%%%%%%%
\section{Навчання нейронної мережі за допомогою генетичних алгоритмів}

У контексті проекту нейронні мережі використовують
генетичні алгоритми для навчання замість використання
такого типового методу, як градієнтний спуск.
Кожна особа у популяції представляє собою набір ваг та зсувів,
які кодують геном та фенотип організму.
Продуктивність роботи цієї нейронної мережі, як механізм
реагування організму на навколишнє середовище,
є основним об'єктом для оцінки придатності кожної особини.

У даному випадку використання генетичних алгоритмів для
навчання нейроних мереж організмів має декілька переваг:
\begin{enumerate}
  \item \textbf{Глобальний пошук}. Оскільки ГА підтримують популяцію рішень
    і використовують механізм мутації,
    вони менш схильні до зациклення на локальних мінімумах.
  \item На відміну від градієнтних підходів, 
    генетичні алгоритми не потребують знання похідних функції.
  \item ГА забезпечують природний метод 
    кодування і розвитку як архітектури, 
    так і вагових коефіцієнтів у питаннях, 
    де архітектура нейронної мережі 
    (кількість шарів, кількість нейронів на шар і т.д.) 
    також є частиною проблеми оптимізації.
\end{enumerate}

Генетичне представлення нейронних мереж є 
унікальним компонентом цієї роботи. 
<<Геном>> кожної істоти представлений у вигляді списку 
всіх вагових коефіцієнтів нейронної мережі. 
Ці ваги керують поведінкою організму і можуть змінюватися 
в результаті еволюційних процесів
таких як мутації та кросинговеру. 
Кодування дійсними числами в генетичних алгоритмах 
представляє нейронні мережі організмів у вигляді вектора дійсних чисел,
що надає їм значення генома для процесу еволюції.


Така ідея чудово ілюструє можливості 
глибокого навчання та генетичних алгоритмів. 
Глибинне навчання дозволяє простим організмам приймати 
розумні рішення щодо свого оточення. 
Генетичні алгоритми, з іншого боку, 
дозволяють цим організмам еволюціонувати з покоління в покоління, 
зрештою підвищуючи їхній інтелект та адаптивність. 



%%%%%%%%%%%%%%%%%%%%%%%%%%%%%%%%%%%%%%%%%%%%%%%%%%%%%%%%%%%%%%%%%%%%%%%%%%%%%%%
\section{Процес моделювання розвитку простих організмів}

Нейронна мережа отримує вхідні дані про навколишнє середовище, 
а на виході - діяльність організму. 
Повинен існувати певний механізм <<зору>> таких організмів,
що виводив би інформацію про середовище через призму певних
обмежень та факторів.
Кількість таких механізмів для порівняння не є обмеженою.

Функція пристосованості у генетичному алгоритмі, 
яка впливає на виживання та розмноження організму, 
може базуватися на рівні енергії організму та 
співвідношенні спожитої їжі до пройденої відстані. 
Це сприяє еволюції видів, які можуть ефективно добувати їжу, 
витрачаючи при цьому найменшу кількість енергії.
Але можна використовувати і простішу реалізацію функції пристосованості,
як поточний рівень енергії в організмі.

Модель також повинна включає в себе елітизм та усікаючий відбір,
для роботи механізмів рекомбінації та мутації.
Елітизм гарантує, що найкращі особини з одного покоління 
передаються наступному, підтримуючи хороші рішення, 
тоді як усікаючий відбір видаляє частину популяції з найнижчими 
показниками пристосованості перед розмноженням, 
прискорюючи еволюційний процес.

Крім того, модель використовує ряд операторів 
кросинговеру та мутації у генетичному алгоритмі:
SBX рекомбінація, 
арифметичний кросинговер 
та змішаний кросинговер (BLX-0.5) --- 
це методи змішування "геномів" двох батьківських 
організмів для отримання нащадків. 
Мутація використовується для внесення випадкових 
змін у "геном" за допомогою гауссових, 
рівномірних і нерівномірних мутацій. 
Ці різні оператори урізноманітнюють популяцію, 
що дозволяє легше досліджувати простір розв'язків.

Глибинне навчання використовується у проекті для розробки 
"механізму мислення" для базових організмів. 
Глибинні нейронні мережі приймають дані про навколишнє середовище 
як вхідні дані, фільтрують їх через численні взаємопов'язані 
шари для вилучення та вивчення складних властивостей 
і видають вихідні дані, які відповідають діяльності організму. 
Завдяки цьому процесу організм може успішно сприймати навколишнє 
середовище і приймати розумні рішення щодо свого наступного кроку.

За допомогою методу дискової вибірки Пуассона
\cite{bridsonFastPoissonDisk2007}
моделюється поява їжі та організмів в екосистемі. 
Цей алгоритм генерує випадково розподілені точки, 
зберігаючи мінімальну відстань між ними, 
імітуючи природну дисперсію ресурсів та 
істот у середовищі існування.

По суті, модель являє собою синтез глибокого навчання 
(представленого нейронними мережами) 
з еволюційними алгоритмами (представленими генетичним алгоритмом). 
Мета полягає в тому, щоб побудувати базову, 
але надійну модель розвитку організму в часі, 
керуючись принципами виживання, адаптації та еволюції.




%%%%%%%%%%%%%%%%%%%%%%%%%%%%%%%%%%%%%%%%%%%%%%%%%%%%%%%%%%%%%%%%%%%%%%%%%%%%%%%
\hide{
У другому розділі необхідно наводити розв'язання поставленої перед вами 
задачі у теоретичному або аналітичному сенсі (хоча, звісно, все залежить 
від того, яка саме задача перед вами поставлена).

Бажано, щоб кожен пункт завдань, окреслених у вступі, відповідав певному 
розділу або підрозділу у дипломній роботі.
}

%%%%%%%%%%%%%%%%%%%%%%%%%%%%%%%%%%%%%%%%%%%%%%%%%%%%%%%%%%%%%%%%%%%%%%%%%%%%%%%

\chapconclude{\ref{chap:theory}}

Наприкінці розділу знову наводяться коротенькі підсумки.
