%%%% У даний файл додавайте всі необхідні вам додаткові пакети, наприклад...


%%%% Диаграммы
%\usepackage{tikz}                      % !!! невідомий конфлікт з якимось іншим пакетом

% Пакети для кольорових текстів (необхідні для команди \todo)
%\usepackage{xcolor}                     % !!! невідомий конфлікт з якимось іншим пакетом
%\usepackage{colortbl}

\usepackage{euscript}   %ещё один красивый шрифт \EuScript

\usepackage{url}
\usepackage{graphicx}
\usepackage{float}
\usepackage{booktabs}
\usepackage{enumitem}

% \usepackage[bibstyle=gost-numeric,sorting=none]{biblatex}


% force display math for all environements
\everymath{\displaystyle}

\usepackage{witharrows}

% Math stuff
\usepackage{amsmath, amsfonts, mathtools, amsthm, amssymb}
% Fancy script capitals
\usepackage{mathrsfs}
\usepackage{cancel}
% Bold math
\usepackage{bm}

%\usepackage[normalem]{ulem} % для подчёркиваний uline
%\ULdepth = 0.16em % расстояние от линии до текста выше/ниже

% codeblocks (listings)
\usepackage{listings}
\usepackage{color}

\definecolor{mygreen}{rgb}{0,0.6,0}
\definecolor{mygray}{rgb}{0.5,0.5,0.5}
\definecolor{mymauve}{rgb}{0.58,0,0.82}

\lstset{ 
  backgroundcolor=\color{white},   % choose the background color; you must add \usepackage{color} or \usepackage{xcolor}; should come as last argument
  % basicstyle=\footnotesize,        % the size of the fonts that are used for the code
  % basicstyle=\sffamily,
  basicstyle=\fontsize{11}{13}\selectfont\ttfamily,
  breakatwhitespace=false,         % sets if automatic breaks should only happen at whitespace
  breaklines=true,                 % sets automatic line breaking
  % captionpos=b,                    % sets the caption-position to bottom
  commentstyle=\color{mygreen},    % comment style
  % escapeinside={\%*}{*)},          % if you want to add LaTeX within your code
  extendedchars=true,              % lets you use non-ASCII characters; for 8-bits encodings only, does not work with UTF-8
  firstnumber=1,                % start line enumeration with line 1000
  frame=single,	                   % adds a frame around the code
  keepspaces=true,                 % keeps spaces in text, useful for keeping indentation of code (possibly needs columns=flexible)
  keywordstyle=\color{blue},       % keyword style
  % language=Octave,                 % the language of the code
  morekeywords={while, null, endwhile, return, :=, do, times, for, which, each},            % if you want to add more keywords to the set
  % numbers=left,                    % where to put the line-numbers; possible values are (none, left, right)
  % numbersep=5pt,                   % how far the line-numbers are from the code
  % numberstyle=\tiny\color{mygray}, % the style that is used for the line-numbers
  rulecolor=\color{gray},         % if not set, the frame-color may be changed on line-breaks within not-black text (e.g. comments (green here))
  showspaces=false,                % show spaces everywhere adding particular underscores; it overrides 'showstringspaces'
  showstringspaces=false,          % underline spaces within strings only
  showtabs=false,                  % show tabs within strings adding particular underscores
  stepnumber=1,                    % the step between two line-numbers. If it's 1, each line will be numbered
  stringstyle=\color{mymauve},     % string literal style
  tabsize=2,	                   % sets default tabsize to 2 spaces
  title=\lstname                   % show the filename of files included with \lstinputlisting; also try caption instead of title
}
%%%%%%%%%%%%%%%%%%%%%%%

% Todonotes and inline notes in fancy boxes
\usepackage{todonotes}
\usepackage{tcolorbox}

% Figure support https://github.com/gillescastel/inkscape-figures
\usepackage{import}
\usepackage{xifthen}
\usepackage{pdfpages}
\usepackage{transparent}
\newcommand{\incfig}[1]{%
    \def\svgwidth{\columnwidth}
    \import{./figures/}{#1.pdf_tex}
}
